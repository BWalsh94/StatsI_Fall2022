\documentclass[12pt,letterpaper]{article}
\usepackage{graphicx,textcomp}
\usepackage{natbib}
\usepackage{setspace}
\usepackage{fullpage}
\usepackage{color}
\usepackage[reqno]{amsmath}
\usepackage{amsthm}
\usepackage{fancyvrb}
\usepackage{amssymb,enumerate}
\usepackage[all]{xy}
\usepackage{endnotes}
\usepackage{lscape}
\newtheorem{com}{Comment}
\usepackage{float}
\usepackage{hyperref}
\newtheorem{lem} {Lemma}
\newtheorem{prop}{Proposition}
\newtheorem{thm}{Theorem}
\newtheorem{defn}{Definition}
\newtheorem{cor}{Corollary}
\newtheorem{obs}{Observation}
\usepackage[compact]{titlesec}
\usepackage{dcolumn}
\usepackage{tikz}
\usetikzlibrary{arrows}
\usepackage{multirow}
\usepackage{xcolor}
\newcolumntype{.}{D{.}{.}{-1}}
\newcolumntype{d}[1]{D{.}{.}{#1}}
\definecolor{light-gray}{gray}{0.65}
\usepackage{url}
\usepackage{listings}
\usepackage{color}

\definecolor{codegreen}{rgb}{0,0.6,0}
\definecolor{codegray}{rgb}{0.5,0.5,0.5}
\definecolor{codepurple}{rgb}{0.58,0,0.82}
\definecolor{backcolour}{rgb}{0.95,0.95,0.92}

\lstdefinestyle{mystyle}{
	backgroundcolor=\color{backcolour},   
	commentstyle=\color{codegreen},
	keywordstyle=\color{magenta},
	numberstyle=\tiny\color{codegray},
	stringstyle=\color{codepurple},
	basicstyle=\footnotesize,
	breakatwhitespace=false,         
	breaklines=true,                 
	captionpos=b,                    
	keepspaces=true,                 
	numbers=left,                    
	numbersep=5pt,                  
	showspaces=false,                
	showstringspaces=false,
	showtabs=false,                  
	tabsize=2
}
\lstset{style=mystyle}
\newcommand{\Sref}[1]{Section~\ref{#1}}
\newtheorem{hyp}{Hypothesis}

\title{Problem Set 3}
\date{Due: November 20, 2022}
\author{Applied Stats/Quant Methods 1}


\begin{document}
	\maketitle
	\section*{Instructions}
	\begin{itemize}
		\item Please show your work! You may lose points by simply writing in the answer. If the problem requires you to execute commands in \texttt{R}, please include the code you used to get your answers. Please also include the \texttt{.R} file that contains your code. If you are not sure if work needs to be shown for a particular problem, please ask.
		\item Your homework should be submitted electronically on GitHub.
		\item This problem set is due before 23:59 on Sunday November 20, 2022. No late assignments will be accepted.
		\item Total available points for this homework is 80.
	\end{itemize}
	
	\vspace{.25cm}
	
	\noindent In this problem set, you will run several regressions and create an add variable plot (see the lecture slides) in \texttt{R} using the \texttt{incumbents\_subset.csv} dataset. Include all of your code.
	
	\vspace{.5cm}
	\section*{Question 1}
	\vspace{.25cm}
	\noindent We are interested in knowing how the difference in campaign spending between incumbent and challenger affects the incumbent's vote share. \vspace{.25cm}
	\begin{enumerate}
		\item Run a regression where the outcome variable is \texttt{voteshare} and the explanatory variable is \texttt{difflog}.
		
		\-\hspace{0.5cm}\textbf{R code:} difflog$\_$voteshare$\_$regression $\gets$ lm(voteshare $\sim$ difflog, data = Incumbent$\_$data)
		
		\vspace{0.5cm}
		\begin{table}[h!]
			\-\hspace{0.5cm}\textbf{Residuals:}
			\centering
			\begin{tabular}{c c c c c}
				Min & 1Q & Median & 3Q & Max \\
				\\[-1.8ex] 
				\hline \\[-1.8ex]
				-0.26832 & -0.05345 & -0.00377 & 0.04780 & 0.32749\\
			\end{tabular}
		\end{table}
		\vspace{0.5cm}
		\begin{table}[h!]
			\-\hspace{0.5cm}\textbf{Summary Table:}
			\centering
			\begin{tabular}{l | c c c c }
				& Estimate & Standard Error & t-value & Pr\texttt{(>|t|)} \\
				\\[-1.8ex] 
				\hline \\[-1.8ex]
				(Intercept) & 0.579031 & 0.002251 & 257.19 & $<$2e-16*** \\
				difflog & 0.041666 & 0.000968 & 43.04 & $<$2e-16*** \\
			\end{tabular}
		\end{table}
		\vspace{0.5cm}
		\begin{table}[h!]
			\centering
			\begin{tabular}{c c c c c c}
				\\[-1.8ex] 
				Significance Codes: 0'***' & 0.001'**' & 0.01'*' & 0.05'.' & 0.2 '' & 1 \\
			\end{tabular}
		\end{table}
		
		Residual standard error: 0.07867 on 3191 degrees of freedom
		
		Multiple R-squared: 0.3673, Adjusted R-squared: 0.3671 
		
		F-statistic: 1853 on 1 and 3191 DF
		
		p-value: $<$ 2.2e-16
		
		The p-value is significant at \underline{99.9\%}	
		\vspace{1cm}
		\item Make a scatterplot of the two variables and add the regression line. 
		
		\includegraphics[width=0.9\textwidth]{Q1 voteshare-y regressed on difflog-x}
		\item Save the residuals of the model in a separate object.
		
		\-\hspace{0.5cm}\textbf{R code:} Q1residuals$\gets$resid(difflog$\_$voteshare$\_$regression)
		\vspace{1cm}
		\item Write the prediction equation.
		
		\-\hspace{0.5cm}\widehat{y} = 0.579 + 0.042\texttt{(difflog)}
	\end{enumerate}
	
	\newpage
	
	\section*{Question 2}
	\noindent We are interested in knowing how the difference between incumbent and challenger's spending and the vote share of the presidential candidate of the incumbent's party are related.	\vspace{.25cm}
	\begin{enumerate}
		\item Run a regression where the outcome variable is \texttt{presvote} and the explanatory variable is \texttt{difflog}.
		
		\-\hspace{0.5cm}\textbf{R code:} presvote$\_$difflog$\_$regression $\gets$ lm(presvote $\sim$ difflog, data = Incumbent$\_$data)
		
		\vspace{0.5cm}
		\begin{table}[h!]
			\-\hspace{0.5cm}\textbf{Residuals:}
			\centering
			\begin{tabular}{c c c c c}
				Min & 1Q & Median & 3Q & Max \\
				\\[-1.8ex] 
				\hline \\[-1.8ex]
				-0.32196 & -0.07407 & -0.00102 & 0.07151 & 0.42743\\
			\end{tabular}
		\end{table}
		\vspace{0.5cm}
		\begin{table}[h!]
			\-\hspace{0.5cm}\textbf{Summary Table:}
			\centering
			\begin{tabular}{l | c c c c }
				& Estimate & Standard Error & t-value & Pr\texttt{(>|t|)} \\
				\\[-1.8ex] 
				\hline \\[-1.8ex]
				(Intercept) & 0.507583 & 0.003161 & 160.60 & $<$2e-16*** \\
				difflog & 0.023837 & 0.001359 & 17.54 & $<$2e-16*** \\
			\end{tabular}
		\end{table}
		\vspace{0.5cm}
		\begin{table}[h!]
			\centering
			\begin{tabular}{c c c c c c}
				\\[-1.8ex] 
				Significance Codes: 0'***' & 0.001'**' & 0.01'*' & 0.05'.' & 0.2 '' & 1 \\
			\end{tabular}
		\end{table}
		
		Residual standard error: 0.1104 on 3191 degrees of freedom
		
		Multiple R-squared: 0.08795, Adjusted R-squared: 0.08767 
		
		F-statistic: 307.7 on 1 and 3191 DF
		
		p-value: $<$ 2.2e-16
		
		The p-value is significant at \underline{99.9\%}
		
		\vspace{2cm}
		\item Make a scatterplot of the two variables and add the regression line. 	
		
		\includegraphics[width=0.9\textwidth]{Q2 presvote-y regressed on difflog-x}
		\item Save the residuals of the model in a separate object.
		
		\-\hspace{0.5cm}\textbf{R code:} Q2residuals$\gets$resid(difflog$\_$presvote$\_$regression)
		\vspace{1cm}
		\item Write the prediction equation.
		
		\-\hspace{0.5cm}\widehat{y} = 0.508 + 0.024\texttt{(difflog)}
	\end{enumerate}
	
	\newpage	
	\section*{Question 3}
	
	\noindent We are interested in knowing how the vote share of the presidential candidate of the incumbent's party is associated with the incumbent's electoral success.
	\vspace{.25cm}
	\begin{enumerate}
		\item Run a regression where the outcome variable is \texttt{voteshare} and the explanatory variable is \texttt{presvote}.
		
		\-\hspace{0.5cm}\textbf{R code:} voteshare$\_$presvote$\_$regression $\gets$ lm(voteshare $\sim$ presvote, data = Incumbent$\_$data)
		\vspace{0.5cm}
		\begin{table}[h!]
			\-\hspace{0.5cm}\textbf{Residuals:}
			\centering
			\begin{tabular}{c c c c c}
				Min & 1Q & Median & 3Q & Max \\
				\\[-1.8ex] 
				\hline \\[-1.8ex]
				-0.27330 & -0.05888 & 0.00394 & 0.06148 & 0.41365\\
			\end{tabular}
		\end{table}
		\vspace{0.5cm}
		\begin{table}[h!]
			\-\hspace{0.5cm}\textbf{Summary Table:}
			\centering
			\begin{tabular}{l | c c c c }
				& Estimate & Standard Error & t-value & Pr\texttt{(>|t|)} \\
				\\[-1.8ex] 
				\hline \\[-1.8ex]
				(Intercept) & 0.441330 & 0.007599 & 58.08 & $<$2e-16*** \\
				presvote & 0.388018 & 0.013493 & 28.76 & $<$2e-16*** \\
			\end{tabular}
		\end{table}
		\vspace{0.5cm}
		\begin{table}[h!]
			\centering
			\begin{tabular}{c c c c c c}
				\\[-1.8ex] 
				Significance Codes: 0'***' & 0.001'**' & 0.01'*' & 0.05'.' & 0.2 '' & 1 \\
			\end{tabular}
		\end{table}
		
		Residual standard error: 0.08815 on 3191 degrees of freedom
		
		Multiple R-squared: 0.2058, Adjusted R-squared: 0.2056 
		
		F-statistic: 827 on 1 and 3191 DF
		
		p-value: $<$ 2.2e-16
		
		The p-value is significant at \underline{99.9\%}
		\vspace{1cm}
		\item Make a scatterplot of the two variables and add the regression line. 
		
		\includegraphics[width=0.9\textwidth]{Q3 voteshare-y regressed on presvote-x}
		\item Write the prediction equation.
		
		\-\hspace{0.5cm}\widehat{y} = 0.441 + 0.388\texttt{(presvote)}
	\end{enumerate}
	
	
	\newpage	
	\section*{Question 4}
	\noindent The residuals from part (a) tell us how much of the variation in \texttt{voteshare} is $not$ explained by the difference in spending between incumbent and challenger. The residuals in part (b) tell us how much of the variation in \texttt{presvote} is $not$ explained by the difference in spending between incumbent and challenger in the district.
	\begin{enumerate}
		\item Run a regression where the outcome variable is the residuals from Question 1 and the explanatory variable is the residuals from Question 2.
		
		\-\hspace{0.5cm}\textbf{R code:} Q1residuals$\_$regressed$\_$on$\_$Q2residuals $\gets$ lm(Q1residuals $\sim$ Q2residuals)
		\vspace{0.5cm}
		
		\begin{table}[h!]
			\-\hspace{0.5cm}\textbf{Residuals:}
			\centering
			\begin{tabular}{c c c c c}
				Min & 1Q & Median & 3Q & Max \\
				\\[-1.8ex] 
				\hline \\[-1.8ex]
				-0.25928 & -0.04737 & -0.00121 & 0.04618 & 0.33126\\
			\end{tabular}
		\end{table}
		\vspace{0.5cm}
		\begin{table}[h!]
			\-\hspace{0.5cm}\textbf{Summary Table:}
			\centering
			\begin{tabular}{l | c c c c }
				& Estimate & Standard Error & t-value & Pr\texttt{(>|t|)} \\
				\\[-1.8ex] 
				\hline \\[-1.8ex]
				(Intercept) & -4.860e-18 & 1.299e-03 & 0.00 & 1 \\
				Q2 residuals & 2.569e-01 & 1.176e-02 & 21.84 & $<$2e-16*** \\
			\end{tabular}
		\end{table}
		\vspace{0.5cm}
		\begin{table}[h!]
			\centering
			\begin{tabular}{c c c c c c}
				\\[-1.8ex] 
				Significance Codes: 0'***' & 0.001'**' & 0.01'*' & 0.05'.' & 0.2 '' & 1 \\
			\end{tabular}
		\end{table}
		
		Residual standard error: 0.07338 on 3191 degrees of freedom
		
		Multiple R-squared: 0.13, Adjusted R-squared: 0.12
		
		F-statistic: 477 on 1 and 3191 DF
		
		p-value: $<$ 2.2e-16
		
		The p-value is significant at \underline{99.9\%}
		
		\vspace{6cm}
		\item Make a scatterplot of the two residuals and add the regression line. 
		
		\includegraphics[width=0.9\textwidth]{Q4-residuals}
		\item Write the prediction equation.
		
		\-\hspace{0.5cm}\widehat{y} = -4.860e-18 + 2.59e-01\texttt{(Q2residuals)}
	\end{enumerate}
\end{enumerate}

\newpage	

\section*{Question 5}
\noindent What if the incumbent's vote share is affected by both the president's popularity and the difference in spending between incumbent and challenger? 
\begin{enumerate}
	\item Run a regression where the outcome variable is the incumbent's \texttt{voteshare} and the explanatory variables are \texttt{difflog} and \texttt{presvote}.
	
	\-\hspace{0.5cm}\textbf{R code:} voteshare$\_$regressed$\_$on$\_$difflog$\_$and$\_$presvote $\gets$ lm(voteshare $\sim$ difflog + presvote, data = Incumbent$\_$data)
	
	\begin{table}[h!]
		\-\hspace{0.5cm}\textbf{Residuals:}
		\centering
		\begin{tabular}{c c c c c}
			Min & 1Q & Median & 3Q & Max \\
			\\[-1.8ex] 
			\hline \\[-1.8ex]
			-0.25928 & -0.04737 & -0.00121 & 0.04618 & 0.33126\\
		\end{tabular}
	\end{table}
	\vspace{0.5cm}
	\begin{table}[h!]
		\-\hspace{0.5cm}\textbf{Summary Table:}
		\centering
		\begin{tabular}{l | c c c c }
			& Estimate & Standard Error & t-value & Pr\texttt{(>|t|)} \\
			\\[-1.8ex] 
			\hline \\[-1.8ex]
			(Intercept) & 0.4486442 & 0.0063297 & 70.88 & $<$2e-16*** \\
			difflog & 0.0355431 & 0.0009455 & 37.59 & $<$2e-16*** \\
			presvote & 0.2568770 & 0.0117637 & 21.84& $<$2e-16*** \\
		\end{tabular}
	\end{table}
	\vspace{0.5cm}
	\begin{table}[h!]
		\centering
		\begin{tabular}{c c c c c c}
			\\[-1.8ex] 
			Significance Codes: 0'***' & 0.001'**' & 0.01'*' & 0.05'.' & 0.2 '' & 1 \\
		\end{tabular}
	\end{table}
	
	Residual standard error: 0.07339 on 3190 degrees of freedom
	
	Multiple R-squared: 0.4496, Adjusted R-squared: 0.4493
	
	F-statistic: 1303 on 2 and 3190 DF
	
	p-value: $<$ 2.2e-16
	
	The p-value is significant at \underline{99.9\%}
	\vspace{0.5cm}
	
	\item Write the prediction equation.	
	
	\-\hspace{0.5cm}\widehat{y} = 0.449 + 0.388\texttt{(difflog)} + 0.257\texttt{(presvote)}
	
	\item What is it in this output that is identical to the output in Question 4? Why do you think this is the case?
	
	\vspace{0.5cm}
	The residuals in both Question 4 and Question 5 are identical. I believe this to be so because both are examining different sides of the same coin; i.e., the same variables: \texttt{presvote}, \texttt{difflog}, and \texttt{voteshare}.
		
	\end{enumerate}
	
	
	
	
\end{document}